\documentclass[aspectratio=169]{ctexbeamer}
\usepackage[english]{babel}
\usepackage[authormarkup=none]{changes}
\usefonttheme{professionalfonts}   % 不改变数学字体
\usepackage{amsmath,amssymb}       % 数学包
\usepackage{xcolor}
\usepackage{tikz}
\usetikzlibrary{calc}
\usepackage{graphicx}
\useoutertheme[width=3\baselineskip,right]{sidebar}
% 目录标数字
\setbeamertemplate{section in toc}[sections numbered] 
% 无序列表用实心点
\setbeamertemplate{itemize item}{$\bullet$}
% 设置每页标题格式
\setbeamertemplate{frametitle}
  {\vspace{-0.5cm}
   \insertframetitle
   \vspace{-0.5cm}}
% 去掉下面没用的导航条
\setbeamertemplate{navigation symbols}{}
% 设置页脚格式
\makeatother
\setbeamertemplate{footline}
{
  \leavevmode%
  \hbox{%
  \begin{beamercolorbox}[wd=.4\paperwidth,ht=2.25ex,dp=1ex,center]{author in head/foot}%
    \usebeamerfont{author in head/foot}\insertshortauthor
  \end{beamercolorbox}

  \begin{beamercolorbox}[wd=.6\paperwidth,ht=2.25ex,dp=1ex,center]{title in head/foot}%
    \usebeamerfont{title in head/foot}\insertshorttitle\hspace*{13em}
    \insertframenumber{} / \inserttotalframenumber\hspace*{0ex}
  \end{beamercolorbox}}

  \vskip0pt%
}
\makeatletter

% 定义颜色
%\definecolor{alizarin}{rgb}{0.82, 0.1, 0.26} % 红色
%\definecolor{DarkFern}{HTML}{407428} % 绿色
%\colorlet{main}{DarkFern!100!white} % 第一种设置方法
%\colorlet{main}{red!70!black} % 第二种设置方法
\definecolor{bistre}{rgb}{0.24, 0.17, 0.12} % 黑色
\definecolor{mygrey}{rgb}{0.52, 0.52, 0.51} % 灰色
\colorlet{main}{green!50!black}
\colorlet{text}{bistre!100!white}

% 不同元素指定不同颜色,fg是本身颜色,bg是背景颜色,!num!改变数值提供渐变色
\setbeamercolor{title}{fg=main}
\setbeamercolor{frametitle}{fg=main}
\setbeamercolor{section in toc}{fg=text}
\setbeamercolor{normal text}{fg=text}
\setbeamercolor{block title}{fg=main,bg=mygrey!14!white}
\setbeamercolor{block body}{fg=black,bg=mygrey!10!white}
\setbeamercolor{qed symbol}{fg=main} % 证明结束后的框颜色
\setbeamercolor{math text}{fg=black}
% 设置页脚对应位置颜色
\setbeamercolor{author in head/foot}{fg=black, bg=mygrey!5!white}
\setbeamercolor{title in head/foot}{fg=black, bg=mygrey!5!white}
\setbeamercolor{structure}{fg=main, bg=mygrey!10!white} % 设置sidebar颜色

% 左右页间距的排版
\def\swidth{2.3cm}
\setbeamersize{sidebar width right=\swidth}
\setbeamersize{sidebar width left=\swidth}
\setbeamerfont{title in sidebar}{size=\scriptsize}
\setbeamerfont{section in sidebar}{size=\tiny}


%-------------------正文-------------------------%

\author{沉石鱼惊旋}
\title{MX-J27 Solution}
\date{October 25, 2025}

\begin{document}

\frame[plain]{\titlepage}

\begin{frame}
  \frametitle{Outline}
  \tableofcontents
\end{frame}

\section{T1 分块}

\frame{\frametitle{Outline}\tableofcontents[currentsection]}

\begin{frame}
  \frametitle{题意}

  \begin{itemize}
    \item <1 - > 给定 $n$,询问有多少个 $1\leq x\leq n$ 满足 $\lfloor\sqrt{x}\rfloor$ 是 $x$ 的因子。
    \item <2 - > $q$ 次询问,$1\leq n\leq 10^{18}$,$1\leq q\leq 10^5$。
  \end{itemize}

\end{frame}

\begin{frame}
  \frametitle{部分分 40}

  \begin{itemize}
    \item <1 - > 对于一次询问,我们肯定可以直接暴力遍历所有的 $1\leq x\leq n$ 进行判断。
    \item <2 - > 可以预处理,先把 $1$ 到 $10^6$ 都扫一遍,然后可以知道每个 $1\leq n\leq 10^6$ 的答案。
    \item <3 - > 把答案存下来,每次询问即可 $\mathcal O(1)$ 回答。
    \item <4 - > 时间复杂度 $\mathcal O(n+q)$。
  \end{itemize}

\end{frame}

\begin{frame}
  \frametitle{部分分 80}

  \begin{itemize}
    \item <1 - > 这一档分虽然 $\mathcal O(n)$ 做不了了,但是 $\mathcal O(\sqrt n)$ 是可行的。
    \item <2 - > 令 $k=\lfloor\sqrt{x}\rfloor$,我们可以枚举 $k$。
    \item <3 - > 对于一个 $k$,考虑它会给哪些 $x$ 产生贡献。
    \item <4 - > 注意到一定是形如 $x=k(k+c)$ 的形式才会被 $k$ 带来贡献。
    \item <5 - > 具体的,由于 $k(k+3)=k^2+3k$ 而 $(k+1)^2=k^2+2k+1$,因为有 $k\geq 1$ 所以 $k(k+3)\geq (k+1)^2$。
  \end{itemize}

\end{frame}

\begin{frame}
  \frametitle{部分分 80}

  \begin{itemize}
    \item <1 - > 也就是说,$\lfloor\sqrt{k(k+3)}\rfloor\neq k$。
    \item <2 - > 因此,枚举 $k$,计算多少个 $x=k^2,x=k(k+1),x=k(k+2)$ 在 $[1,n]$ 之间。
    \item <3 - > 时间复杂度 $\mathcal O(q\sqrt n)$。
  \end{itemize}

\end{frame}

\begin{frame}
  \frametitle{满分 100}

  \begin{itemize}
    \item <1 - > 80 分的做法,存在一个关键性质:只有 $x=k^2,x=k(k+1),x=k(k+2)$ 这三类形式的 $x$。
    \item <2 - > 而显然,如果存在 $x=k^2$,则 $[1,n]$ 一定存在 $x=(k-1)^2$。其他两种同理。
    \item <3 - > 因此,我们只关心最大的 $x=k^2$ 的 $k$ 是多少。剩下的 $1\leq k'\leq k$ 都一定合法。
  \end{itemize}

\end{frame}

\begin{frame}
  \frametitle{满分 100}

  \begin{itemize}
    \item <1 - > 直接计算 $m=\sqrt{n}$,则形如 $x=k^2$ 的有 $m$ 个。
    \item <2 - > 同理可以计算出 $x=k(k+1)$ 和 $x=k(k+2)$ 形式的数的个数。
    \item <3 - > 具体实现在 $m$ 周围往下枚举若干个,即可知道精确的 $x=k(k+1)$ 的数的个数,或使用二分搜索同理。
    \item <4 - > 实现的时候,如果需要开根号,想要调用 $\textcolor{blue}{\texttt{sqrt(n)}}$,需要注意,对于 $\textcolor{blue}{\texttt{long long}}$ 类型应该使用 $\textcolor{blue}{\texttt{sqrtl(n)}}$,否则会产生精度误差导致挂分。这是因为 $\textcolor{blue}{\texttt{sqrt()}}$ 只在 $\textcolor{blue}{\texttt{double}}$ 精度范围内,而 $\textcolor{blue}{\texttt{sqrtl()}}$ 在 $\textcolor{blue}{\texttt{long double}}$ 精度范围内。$\textcolor{blue}{\texttt{double}}$ 的精度范围内无法精确表示 $\textcolor{blue}{\texttt{long long}}$ 类型。
    \item <5 - > 依据实现时间复杂度为 $\mathcal O(q)$ 或 $\mathcal O(q\log n)$。
  \end{itemize}

\end{frame}

\section{T2 转换}

\frame{\frametitle{Outline}\tableofcontents[currentsection]}

\begin{frame}
  \frametitle{题意}

  \begin{itemize}
    \item <1 - > 给定运算表达式,计算表达式返回值的\textbf{类型}。具体运算规则与 C++ 运算规则相同。
    \item <2 - > 保证类型只在 \{\textcolor{blue}{\texttt{char}},\textcolor{blue}{\texttt{bool}},\textcolor{blue}{\texttt{int}},\textcolor{blue}{\texttt{longlong}},\textcolor{blue}{\texttt{float}},\textcolor{blue}{\texttt{double}}\} 中;保证运算符只在 \{\textcolor{blue}{\texttt{+}},\textcolor{blue}{\texttt{*}},\textcolor{blue}{\texttt{,}}\} 中。
    \item <3 - > $1\leq n\leq 10^5$,$n$ 是表达式中的运算符数量。
  \end{itemize}

\end{frame}

\begin{frame}
  \frametitle{部分分 30}

  \begin{itemize}
    \item <1 - > 暴力模拟运算即可。每一次找到优先级最高的运算符计算。
    \item <2 - > 时间复杂度 $\mathcal O(n^2)$。
  \end{itemize}

\end{frame}

\begin{frame}
  \frametitle{特殊性质}

  \begin{itemize}
    \item <1 - > 特殊性质都是简化模拟操作的,以及防止写挂爆零的,实际上没有太大启发。
  \end{itemize}

\end{frame}

\begin{frame}
  \frametitle{满分 100}

  \begin{itemize}
    \item <1 - > 不妨先对字符串做一些简化。
    \item <2 - > 因为涉及逗号运算只看最后一项,所以可以找到最后一次出现的逗号在位置 $p$,将 $s$ 截断为 $s\gets s[p+1:n]$。
    \item <3 - > 类型太长了,长度也不统一。我们只取他们的首字母表示。
  \end{itemize}

\end{frame}

\begin{frame}
  \frametitle{满分 100}

  \begin{itemize}
    \item <1 - > 考虑表达式计算的做法。
    \item <2 - > 我们维护一个栈。栈里面存的是若干个类型和操作符。
    \item <3 - > 由于乘法优先级高,所以先做一遍乘法计算。
    \item <4 - > 我们把元素压入栈的时候,如果目前栈顶是乘号,再弹出栈顶第二个元素进行计算,把计算结果压回栈内。
    \item <5 - > 做完上述操作之后,重新扫一遍栈,再做加法运算即可。
  \end{itemize}

\end{frame}

\begin{frame}
  \frametitle{满分 100}

  \begin{itemize}
    \item <1 - > 对于操作符的计算,直接模拟即可。
    \item <2 - > 先把 \textcolor{blue}{\texttt{bool}} 和 \textcolor{blue}{\texttt{char}} 全部改成 \textcolor{blue}{\texttt{int}}。
    \item <3 - > 如果操作同时含有浮点类和整数类,返回那个浮点类的类型。
    \item <4 - > 否则,此时两个数类型(浮点 or 整数)一样,返回占用字节数更多的那个。
  \end{itemize}

\end{frame}

\begin{frame}
  \frametitle{满分 100}

  \begin{itemize}
    \item <1 - > 直接模拟这个压栈弹栈的时间复杂度就是 $\mathcal O(n)$ 的。
    \item <2 - > 另外本题可能出现 \textcolor{blue}{\texttt{expr,bool}} 的情况,所以你\textbf{不应该}在一开始就把所有 \textcolor{blue}{\texttt{bool}} 和 \textcolor{blue}{\texttt{char}} 全部改成 \textcolor{blue}{\texttt{int}}。
  \end{itemize}

\end{frame}

\begin{frame}
  \frametitle{还没完}

  \begin{itemize}
    \item <1 - > 有一个更巧妙的观察:乘法和加法无非是运算顺序不一样,但是具体的计算返回类型是一样的。
    \item <2 - > 也就是我们根本不需要区分乘法和加法,全部看为加法进行操作即可,不需要扫两次。
    \item <3 - > 以及,我们可以看作是,每个类型有一个权值,每次取权值最大的。这些运算符全部等价于取 $\max$。
    \item <4 - > 当然作为一道 T2 级别的模拟题,怎么写都可以,在此只是讲一些实现的小技巧。
  \end{itemize}

\end{frame}

\section{T3 旋律}

\frame{\frametitle{Outline}\tableofcontents[currentsection]}

\begin{frame}
  \frametitle{题意}

  \begin{itemize}
    \item <1 - > 给定长度为 $a_1,a_2,\dots,a_n$,选出一个非空子序列 $b_1,b_2,\dots,b_m$,最大化 $mk-(\max\{b\}-\min\{b\})$。
    \item <2 - > $1\leq n\leq 10^5$,$1\leq a_i,k\leq 10^8$。
  \end{itemize}

\end{frame}

\begin{frame}
  \frametitle{部分分 20}

  \begin{itemize}
    \item <1 - > 枚举子序列,依据实现可以做到 $\mathcal O(2^n+n\log n)$ 或 $\mathcal O(2^nn+n\log n)$。
  \end{itemize}

\end{frame}

\begin{frame}
  \frametitle{特殊性质}

  \begin{itemize}
    \item <1 - > 特殊性质 A,比较公差 $d$ 和 $k$,决策是只选一种数,还是全选。
    \item <2 - > 特殊性质 B,可以证明全选是最优解,因为我们加一个数有 $10^8$ 的收益,但是极差的增量不超过 $10^8$。
    \item <3 - > 特殊性质 C,引导选手往正解方向思考,无较优的简单做法。
  \end{itemize}

\end{frame}

\begin{frame}
  \frametitle{部分分 55}

  \begin{itemize}
    \item <1 - > 由于和极差相关,想到把 $a$ 排序,排序后可以证明我们一定选连续区间,则答案就是最大的 $(r-l+1)k-(a_r-a_l)$。
    \item <2 - > 固定 $l$,枚举 $r$,则复杂度为 $\mathcal O(n^2)$。
  \end{itemize}
\end{frame}

\begin{frame}
  \frametitle{满分 100(算法 1)}
  \begin{itemize}
    \item <1 - > 我们把上面的式子 $(r-l+1)k-(a_r-a_l)$ 打开。
    \item <2 - > 得到 $k+(rk-a_r)-(lk-a_l)$。
    \item <3 - > 这个形式告诉我们,我们只需要知道最大和最小的 $ik-a_i$。
    \item <4 - > 排序后扫一遍,存一下当前最小的 $\mathrm{Min}=ik-a_i$,将答案 $\mathrm{Ans}$ 和目前的 $jk-a_j-\mathrm{Min}+k$ 取最大值。
  \end{itemize}
\end{frame}

\begin{frame}
  \frametitle{满分 100(算法 2)}
  \begin{itemize}
    \item <1 - > 对增量考虑。
    \item <2 - > 对 $a$ 排序之后,我们每一次极差的增量是 $a_i-a_{i-1}$,但是可以获得 $k$ 的收益,所以是 $k-(a_i-a_{i-1})$。
    \item <3 - > 我们其实就是要找到一个最大的区间,使得区间里的增量之和最大。
    \item <4 - > 这是典型的最大子段和问题,直接贪心或者线性 DP 均可。
  \end{itemize}
\end{frame}

\begin{frame}
  \frametitle{满分 100}
  \begin{itemize}
    \item <1 - > 两种做法的时间复杂度都是 $\mathcal O(n\log n)$,瓶颈在排序上。
    \item <2 - > 本题几乎不存在正常写代码过了样例挂了的。存在部分选手因为拼暴力暴力写错了而挂分,正式赛场上的对拍是很重要的!
  \end{itemize}
\end{frame}

\section{T4 点灯}

\frame{\frametitle{Outline}\tableofcontents[currentsection]}

\begin{frame}
  \frametitle{题意}

  \begin{itemize}
    \item <1 - > 给定 $n$ 个点 $m$ 条边的无向图,每条边在特定的一段\textbf{后缀}时间 $[w,+\infty)$ 激活。
    \item <2 - > 初始在点 $1$ 有\textbf{充分多}的人,之后每个时刻所有人\textbf{必须}通过一条已经被激活的边移动到相邻点。
    \item <3 - > 求最早时刻使得所有点都有人,或报告无解。
    \item <4 - > $2\leq n\leq 2.5\times 10^4$,$n-1\leq m\leq 5\times 10^4$,$1\leq w\leq 10^9$。
    \item <5 - > 保证无自环无重边,保证所有边激活后图是连通图。
  \end{itemize}

\end{frame}

\begin{frame}
  \frametitle{部分分 8 \& 特殊性质 A}

  \begin{itemize}
    \item <1 - > 观察数据范围,此时满足 $n\leq 10$ 和 $m\leq 20$。
    \item <2 - > 直接维护每个时刻的可达点,模拟题意。
  \end{itemize}

\end{frame}

\begin{frame}
  \frametitle{特殊性质 B}

  \begin{itemize}
    \item <1 - > 如果 $w\neq 1$,则无解,因为时刻一无法进行任何移动。
    \item <2 - > 考虑 $w=1$,此时即为初始图全部连通。
    \item <3 - > 注意到一个点 $u$,如果在时刻 $t$ 从 $p$ 走过来访问了,则可以反复在 $u,p$ 之间移动,使得在所有 $t+2k$ 的时刻都访问到 $u$。
    \item <4 - > 这启发我们记录每个点在奇数时刻和偶数时刻最早抵达的时间。一个点拆成两个点代表奇偶,直接跑 BFS 即可。
    \item <5 - > 枚举最后答案是奇数还是偶数,把所有点对应的访问时间取最大的那个。
    \item <6 - > 如果奇数和偶数两种情况,都存在一个点无法在对应类型的时刻到达,也是无解。
  \end{itemize}

\end{frame}

\begin{frame}
  \frametitle{特殊性质 C}

  \begin{itemize}
    \item <1 - > 防止你无解情况没判干净或者不会判留的分。
  \end{itemize}

\end{frame}

\begin{frame}
  \frametitle{满分 100(算法 1)}

  \begin{itemize}
    \item <1 - > 沿着特殊性质 B 继续思考。唯一的区别就是 $w$ 不同。
    \item <2 - > 然而考虑这个在一条边反复横跳的过程,我们可以一直等到这个点开通了再走过去。
    \item <3 - > 也就是若点 $u$ 在 $t$ 时刻抵达了,若到邻居 $v$ 的边激活时间为 $w$,若 $w\leq t+1$ 则 $t+1$ 就可以到 $v$,否则最早抵达 $v$ 的时间是 $\geq w$ 的第一个和 $t$ 奇偶不同的时刻。
    \item <4 - > 使用 Dijkstra 维护上述过程。
  \end{itemize}

\end{frame}

\begin{frame}
  \frametitle{满分 100(算法 2)}

  \begin{itemize}
    \item <1 - > 我们把边存下来,按 $w$ 排序。注意一条边 $(u,v,w)$ 可能要拆成 2 条边,多加一条 $(u,v,w+1)$。
    \item <2 - > 我们一边做 BFS,一边随着目前的时刻激活边。
    \item <3 - > 激活的转移是基本同上的,但是由于我们只关注被激活的那些边,所以不存在说从时刻 $t$ 突然跳到 $w$ 的情况。一次 BFS 中的时刻都是连续的。
    \item <4 - > 我们把连续时刻的做完之后,再重新做 BFS,从新的时刻开始,再同上跑一遍。直到跑完了所有边没出答案就是无解,或者出了答案直接结束搜索。
    \item <5 - > 简单来说,这个算法就是只关心连续的有用时刻段,每个段分别跑 BFS。这个算法是在 Junior 组考纲内的。
  \end{itemize}

\end{frame}

\begin{frame}
  \frametitle{满分 100}

  \begin{itemize}
    \item <1 - > 如果是第一种算法,实现可以不显式建图以降低常数因子对程序效率的影响。
    \item <2 - > 另外本题 $o=0$ 的情况下,答案如果是无解,不应该输出 $-1\times 0$,而应该输出 $-1$,存在部分选手没有读清题意在这里挂分的情况。
  \end{itemize}

\end{frame}


\section{鲜花 \& 抽奖}

\frame{\frametitle{Outline}\tableofcontents[currentsection]}

\begin{frame}
  \frametitle{鲜花 \& 抽奖}

  % \includegraphics[
  %   width=0.3\textwidth,
  %   trim=0pt 0pt 775pt 0pt,
  %   clip
  % ]{csp2025.png}

  % \includegraphics[
  %   width=0.3\textwidth,
  %   trim=580pt 0pt 695pt 0pt,
  %   clip
  % ]{csp2025.png}

  % 祝大家 CSP2025 RP++,考出自己满意的成绩,不负韶光!

  % \includegraphics[
  %   width=0.3\textwidth,
  %   trim=665pt 0pt 610pt 0pt,
  %   clip
  % ]{csp2025.png}

  % \includegraphics[
  %   width=0.3\textwidth,
  %   trim=750pt 0pt 0pt 0pt,
  %   clip
  % ]{csp2025.png}
  \includegraphics[height=2cm,trim=0pt 0pt 775pt 0pt,clip]{csp2025.png}
  % \includegraphics[height=2cm,trim=580pt 0pt 695pt 0pt,clip]{csp2025.png}%
  % \raisebox{0.8cm}{\parbox[c][1.5cm][c]{0.4\textwidth}{\centering 祝大家 CSP2025 RP++,考出自己满意的成绩,不负韶光!}}
  % \includegraphics[height=2cm,trim=665pt 0pt 610pt 0pt,clip]{csp2025.png}
  \begin{tikzpicture}[baseline=(current bounding box.center)]
  \node (L) at (0,0) {\includegraphics[height=2cm,trim=580pt 0pt 695pt 0pt,clip]{csp2025.png}};
  \node (R) at (7,0) {\includegraphics[height=2cm,trim=665pt 0pt 610pt 0pt,clip]{csp2025.png}};
  \node[text width=3cm, align=center] at ($(L.east)!0.5!(R.west)$) {祝大家 CSP2025 RP++,考出自己满意的成绩,不负韶光!};  
  \end{tikzpicture}
  \includegraphics[height=2cm,trim=750pt 0pt 0pt 0pt,clip]{csp2025.png}
\end{frame}

\end{document}